\documentclass[nobib]{tufte-handout}

\title{nbinteract: Generate Interactive Web Pages From Jupyter Notebooks}

\author[Samuel Lau]{Samuel Lau}

%\date{28 March 2010} % without \date command, current date is supplied

%\geometry{showframe} % display margins for debugging page layout

\usepackage[
  style=numeric,
  citestyle=numeric
]{biblatex}
\addbibresource{thesis.bib}

\usepackage{graphicx} % allow embedded images
  \setkeys{Gin}{width=\linewidth,totalheight=\textheight,keepaspectratio}
  \graphicspath{{graphics/}} % set of paths to search for images
\usepackage{amsmath}  % extended mathematics
\usepackage{booktabs} % book-quality tables
\usepackage{units}    % non-stacked fractions and better unit spacing
\usepackage{multicol} % multiple column layout facilities
\usepackage{lipsum}   % filler text
\usepackage{fancyvrb} % extended verbatim environments
  \fvset{fontsize=\normalsize}% default font size for fancy-verbatim environments

% Standardize command font styles and environments
\newcommand{\doccmd}[1]{\texttt{\textbackslash#1}}% command name -- adds backslash automatically
\newcommand{\docopt}[1]{\ensuremath{\langle}\textrm{\textit{#1}}\ensuremath{\rangle}}% optional command argument
\newcommand{\docarg}[1]{\textrm{\textit{#1}}}% (required) command argument
\newcommand{\docenv}[1]{\textsf{#1}}% environment name
\newcommand{\docpkg}[1]{\texttt{#1}}% package name
\newcommand{\doccls}[1]{\texttt{#1}}% document class name
\newcommand{\docclsopt}[1]{\texttt{#1}}% document class option name
\newenvironment{docspec}{\begin{quote}\noindent}{\end{quote}}% command specification environment

% Used for inline code snippets.
\newcommand{\code}[1]{\texttt{#1}}

\begin{document}

\maketitle% this prints the handout title, author, and date

\begin{abstract}
\noindent
Abstract:
\end{abstract}

\section{Introduction} % (fold)
\label{sec:introduction}

Jupyter notebooks provide a popular document format for authoring, executing,
and publishing code alongside analysis \cite{thomas_jupyter_2016}. Although
Jupyter notebooks were originally designed for use in scientific workflows for
data preparation and analysis, they are becoming an increasingly common choice
for university courses---a survey in 2016 reported that over one hundred
courses across multiple countries use Jupyter in their course content
\cite{hamrick_2016_2016}.

An increasing number of universities now offer data science courses, many of
which use Jupyter because of its broad adoption for data analysis workflows in
both academia and industry. These courses often use Jupyter notebooks as the
preferred medium for homeworks, labs, projects, and lectures. UC Berkeley's
flagship data science courses, for example, use Jupyter for all of these course
components and have even writen their course textbooks in Jupyter notebooks.

As a web technology, Jupyter notebooks also provide a platform for interaction
authoring. For example, the popular \code{ipywidgets} Python library allows
users to create web-based user interfaces to interact with arbitrary Python
functions. Users can create these interfaces using Python directly in the
notebook environment instead of having to use HTML and Javascript,
significantly lowering the overhead usually needed to create these interfaces
\cite{_jupyter-widgets/ipywidgets_}. This ease-of-use encourages instructors
and researchers to create interactive explanations of their work.

Unfortunately, it is difficult to share these interactive notebooks with a
broad audience. Sharing the notebook file itself retains full interactivity but
requires viewers to have Jupyter, Python, and all other packages used in the
notebook installed on their own machines. The freely available Binder service
circumvents this by hosting notebook servers that come pre-packaged with
necessary software. However, both of these options still require viewers to
have prior familiarity with the Jupyter environment, making them less suitable
for use with non-technical viewers. Authors can convert a Jupyter notebook to a
static HTML document and host the document as a publicly-accessible web page.
However, this method does not preserve the interactive elements of the
notebook; the resulting web page only contains text and images.

\code{nbinteract} is a Python package that allows authors to convert Jupyter
notebooks into interactive HTML pages. The interactive elements can use
arbitrary Python code to generate output, including Python libraries that use C
extensions (e.g. \code{numpy} and \code{pandas}) and libraries that create
images (e.g. \code{matplotlib}). The resulting web pages can be used by anyone
with a modern web browser even if the viewer does not have Python or Jupyter
installed on their computer. The \code{nbinteract} package also includes
specialized methods for interactive plots designed for fast interaction
prototyping in the notebook and smooth interaction on static HTML web pages.
We discuss the design of the package, its features and limitations, and its
implications for interaction authoring and sharing.

% section introduction (end)

\section{Related Work} % (fold)
\label{sec:related_work}

\subsection{Jupyter Technologies} % (fold)
\label{sub:jupyter_technologies}

The Jupyter notebook platform allows authoring and editing code, images, and
written explanations together in a single document composed of multiple cells.
The platform is composed of two main components. It includes a frontend---a
web-based authoring environment that users open in their web browsers. The
frontend connects to a Jupyter kernel, a process on the users' computers that
runs code and returns the output to the frontend to display
\cite{thomas_jupyter_2016}.

The \code{ipywidgets} library makes use of Jupyter's web-based frontend to
create interactive elements directly in the notebook. The library includes
Python functions that produce HTML and Javascript when called to create
widgets. When a user interacts with a widget---selecting an option from a
dropdown menu, for example---the \code{ipywidgets} library executes
user-defined Python functions on the Jupyter kernel and renders the result in
the cell \cite{_jupyter-widgets/ipywidgets_}. A number of other specialized
libraries are built on top of \code{ipywidgets}, such as the interactive
plotting library \code{bqplot} \cite{_bqplot_2018} and the molecular
visualization library \code{nglview} \cite{_arose/nglview_}.

Jupyter notebooks use the \code{nbconvert} tool to convert between notebook
formats. \code{nbconvert} also allows notebooks to be converted to static HTML
pages \cite{_jupyter/nbconvert_}. However, these pages do not retain widget
functionality because they do not have access to a Jupyter kernel by default.

The Binder project hosts ephemeral Jupyter notebook servers as a free service
for the general public. It takes a repository of Jupyter notebooks, starts a
Jupyter frontend and Jupyter kernel, and gives users the ability to run the
notebook over the internet instead of having on their local machines
\cite{_binder_}.

% subsection jupyter_technologies (end)

\subsection{Interaction Authoring in Javascript} % (fold)
\label{sub:interaction_authoring_in_javascript}

Javascript is the most commonly used language to design interactions that run
in a web browser. Because most modern web browsers run Javascript natively,
viewers do not have to install additional software in order to make use of
these interactive elements, a key advantage of the language. A number of
authors use Javascript to create interactive articles
\cite{_explorable_,kunin_seeing_} and textbooks \cite{stark_sticigui_2004}.

There are a number of Javascript libraries that provide higher level
abstractions for interaction creation such as D3 and Tangle
\cite{bostock_d$3$_2011,bret_tangle_2018}. Fundamentally, Javascript libraries
require fluency with aspects of web programming such as Javascript syntax and
the document-object model. This additional requirement makes Javascript more
difficult to use for many data scientists; most data science analysis uses
Python and R.
% subsection interaction_authoring_in_javascript (end)

% section related_work (end)

\newpage

\printbibliography

\end{document}
