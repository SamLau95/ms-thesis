\documentclass[nobib]{tufte-handout}

\title{nbinteract: Generate Interactive Webpages From Jupyter Notebooks}

\author[Samuel Lau]{Samuel Lau}

%\date{28 March 2010} % without \date command, current date is supplied

%\geometry{showframe} % display margins for debugging page layout

\usepackage[
  style=numeric,
  citestyle=numeric
]{biblatex}
\addbibresource{thesis.bib}

\usepackage{graphicx} % allow embedded images
  \setkeys{Gin}{width=\linewidth,totalheight=\textheight,keepaspectratio}
  \graphicspath{{graphics/}} % set of paths to search for images
\usepackage{amsmath}  % extended mathematics
\usepackage{booktabs} % book-quality tables
\usepackage{units}    % non-stacked fractions and better unit spacing
\usepackage{multicol} % multiple column layout facilities
\usepackage{lipsum}   % filler text
\usepackage{fancyvrb} % extended verbatim environments
  \fvset{fontsize=\normalsize}% default font size for fancy-verbatim environments

% Standardize command font styles and environments
\newcommand{\doccmd}[1]{\texttt{\textbackslash#1}}% command name -- adds backslash automatically
\newcommand{\docopt}[1]{\ensuremath{\langle}\textrm{\textit{#1}}\ensuremath{\rangle}}% optional command argument
\newcommand{\docarg}[1]{\textrm{\textit{#1}}}% (required) command argument
\newcommand{\docenv}[1]{\textsf{#1}}% environment name
\newcommand{\docpkg}[1]{\texttt{#1}}% package name
\newcommand{\doccls}[1]{\texttt{#1}}% document class name
\newcommand{\docclsopt}[1]{\texttt{#1}}% document class option name
\newenvironment{docspec}{\begin{quote}\noindent}{\end{quote}}% command specification environment

% Used for inline code snippets.
\newcommand{\code}[1]{\texttt{#1}}

\begin{document}

\maketitle% this prints the handout title, author, and date

\begin{abstract}
\noindent
This is the abstract. It will be filled in after everything else is written.
\end{abstract}

\section{Introduction} % (fold)
\label{sec:introduction}

Jupyter notebooks provide a popular document format for authoring, executing,
and publishing code alongside analysis \cite{Kluyver2016b}. Although Jupyter
notebooks were originally designed for use in scientific workflows for data
preparation and analysis, they are becoming an increasingly common choice for
university courses---a survey in 2016 reported that over one hundred courses
across multiple countries use Jupyter in their course content
\cite{Hamrick2016}.

An increasing number of universities now offer data science courses, many of
which use Jupyter because of its broad adoption for data analysis workflows in
both academia and industry. These courses often use Jupyter notebooks as the
preferred medium for homeworks, labs, projects, and lectures. UC Berkeley's
flagship data science courses, for example, use Jupyter for all of these course
components and even use Jupyter notebooks for their course textbooks.

As a web technology, Jupyter notebooks also provide a platform for interaction
authoring. For example, the popular \code{ipywidgets} Python library allows
users to create web-based user interfaces to interact with arbitrary Python
functions. Users can create these interfaces using Python directly in the
notebook environment instead of having to use HTML and Javascript,
significantly lowering the overhead usually needed to create these interfaces.
This ease-of-use encourages instructors and researchers to create interactive
explanations of their work.

Since \code{ipywidgets} is a Jupyter notebook library, however, in order to
view these interactive explanations users have to run the notebook itself.

% section introduction (end)

\newpage

\printbibliography

\end{document}
